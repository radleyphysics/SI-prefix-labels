\begin{center}
\begin{tabular}{cclr@{}lc} 
 \toprule
 \large Symbol & \large Name & \multicolumn{3}{c}{\large{As a number}} & \large Word \\ 
 \midrule
 \si{\tera} & tera & $\times\num{e12}$ &  1 000 000 000 000 & & trillion \\
 \si{\giga} & giga & $\times\num{e9}$ &  1 000 000 000 & & billion \\
 \si{\mega} & mega & $\times\num{e6}$ &   1 000 000 & & million \\
 \si{\kilo} & kilo & $\times\num{e3}$ &   1 000 & & thousand \\
 & &  \textcolor{gray}{$\times\num{10}^0$} &  \textcolor{gray}{1} & & \textcolor{gray}{one}\\
 \si{\milli} & milli & $\times\num{e-3}$ & 0 & .001 & thousandth \\ 
 \si{\micro} & micro & $\times\num{e-6}$ & 0 & .000 001  & millionth \\
 \si{\nano} & nano & $\times\num{e-9}$ & 0 & .000 000 001  & billionth \\ 
 \si{\pico} & pico & $\times\num{e-12}$ & 0 & .000 000 000 001  & trillionth \\
 \bottomrule
 
\end{tabular}

\vspace{3mm}
\begin{minipage}{14cm}
 \textbf{TIPS } \begin{itemize*} \item Always think back to something familiar e.g.\ $\SI{1}{km} = \SI{1000}{m}$.\\
 \phantom{\textbf{TIPS }} \item For $\times\num{e6}$, think ``move the decimal point 6 places to the right''.\\
 \phantom{\textbf{TIPS\hspace*{0.25em} *}} For $\times\num{e-3}$, think ``move the decimal point 3 places to the left''.
 \end{itemize*}
 
\vspace{2mm}
 \textbf{NB} $\SI{1}{m} = \SI{100}{cm}$ (so $\SI{1}{cm} = \frac{1}{100}~\si{m}$)
  \end{minipage}
 \end{center}